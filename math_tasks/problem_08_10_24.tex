\documentclass{article}

\usepackage[margin=1cm]{geometry}

\begin{document}

Determina el valor de todos los numeros reales $\alpha$ tal que, para todo entero positivo $n$, se cumple que:
\[
    \lfloor \alpha \rfloor + \lfloor 2\alpha \rfloor + \ldots + \lfloor n\alpha \rfloor
\]

es un multiplo de $n$.
(Nota: $\lfloor x \rfloor$ equivale al mayor entero menor o igual a $x$. Por ejemplo, $\lfloor 3.14 \rfloor = 3$,
$\lfloor -2.5 \rfloor = -3$ y $\lfloor 7 \rfloor = 7$.)
(Nota: 0 es multiplo de cualquier entero positivo.)
\begin{enumerate}
    \item Demuestra que $\alpha = 0$ es una soluci\'on.
    \item Demuestra que $\alpha = 1$ no es una soluci\'on.
    \item Demuestra que $\alpha = 2$ es una soluci\'on.
    \item Demuestra que $\alpha = 2k$, para todo $k$ entero, es una soluci\'on(es decir, que todo entero par es soluci\'on).
    \item Demuestra que $\alpha = 1/2$ no es una soluci\'on.
    \item Demuestra que todo $\alpha \in (0, 1)$ no es una soluci\'on.
    \item Demuestra igualmente, que todo $\alpha \in (-1, 0)$ no es una soluci\'on.
    \item Para terminar, considera $\alpha = 2k + \beta$, donde $k$ es un entero y $\beta \in (-1, 1)$, primero demuestra que todo n\'umero real es representable de esta forma.
    \item Demuestra que $\alpha = 2k + \beta$ nunca es soluci\'on, a menos que $\beta = 0$.
\end{enumerate}
\end{document}
